\documentclass[12pt]{article}

\usepackage{mathpazo}
\usepackage[francais]{babel}
\usepackage[T1]{fontenc}
\usepackage[utf8]{inputenc}
\usepackage{array, multirow, tabularx}
\usepackage{graphicx}
\usepackage{multicol}
\usepackage{eurosym}

\begin{document}

\title{Règlement intérieur du Rézoléo}
\author{}
\date{11/07/2017}
\maketitle

\textbf{\large{Préambule : \\}}

\noindent Le présent réglement a été ratifé par les représentants du Bureau le  29/08/2017.

\newpage

\tableofcontents

\newpage

\section{Modalité d'adhésion et cotisation}
	\subsection{Cotisations}

		Pour devenir membre connecté de l'association, la cotisation forfaitaire est de 80\euro \ pour une année. Dans le cas d'une adhésion ou d'un départ au milieu de l'année scolaire, le montant de la cotisation due pour l'année est calculée au pro rata de la durée de l'adhésion, à raison de 8\euro \ par mois, arrondie au nombre de mois supérieur. L'association peut, sur approbation du bureau, décider d'exonérer d'adhésion des personnes restant moins d'un mois.

		Les membres d'honneur et non connectés sont exonérés de cotisation.

    \subsection{Moyens de payement}

		Les seuls moyens de paiement acceptés par l'association sont :

			\begin{itemize}
		    	\item[\textbullet] Les chèques
		    	\item[\textbullet] Virement bancaire
			\item[\textbullet] Les espèces dans le cas d'un abonnement annuel 
		    \end{itemize}
		    

    \subsection{Adhésion}

		Le Bureau peut octroyer à certains membres actifs le droit d’effectuer régulièrement des adhésions. Ceux-ci doivent être conscients qu’ils sont responsables du bon remplissage de la fiche d’adhésion, ainsi que de la bonne réception par l’association de la cotisation de l’adhérent, ainsi que du paiement d’éventuels services supplémentaires.

    \subsection{Accès au réseau et services mis à disposition}

		Un réseau interne à la résidence ainsi que les infrastructures de celui-ci et un certain nombre de services sont mis à disposition des membres de l'association. Les modalités d'utilisation de ces ressources seront déterminées au cas par cas par le bureau.

		\begin{itemize}
			 \item[\textbullet] Une connexion à Internet pour les membres connectés disposant de machines hébergées à la résidence;
			 \item[\textbullet] La possibilité d'héberger des sites internet;
			 \item[\textbullet] Une boite de courriels appartenant à l'association et listes de diffusion
			 \item[\textbullet] Le prêt d'une ou plusieurs machines virtuelles
		 \end{itemize}
		  %//autres services (on peut peut-être faire un tableau)

		Néanmoins, un ou plusieurs de ces services peuvent être indisponibles à tout moment, que ce soit pour des raisons techniques ou à l'occasion d'évènements spéciaux, et ce sans aucun recours possible de la part de l'adhérent.

    \subsection{Club et associations}

		Les clubs et associations présents à la Résidence Léonard de Vinci et à l'Ecole Centrale de Lille peuvent obtenir à titre gracieux un accès aux services habituellement réservés aux membres connectés

\section{Responsabilité de l'Association}

    \subsection{Usage des droits conférés à un membre actif}

		Tout usage par un membre actif de ses droits dans une situation inappropriée, ou hors du cadre de ses missions pour l’association, est interdit. Un tel usage pourra être sanctionné sans délai par un retrait des attributions en question, dans le respect des statuts et du présent règlement. Selon la nature de l’infraction, des mesures juridiques pourront être prises par l’association.

    \subsection{Responsabilité des membres actifs}

		L’association assure le travail de ses membres actifs, leur matériel, et leur personne, dans le cadre de l’exercice de leurs fonctions. Elle prend la responsabilité de toute altération de la connectivité à Internet ou dommage aux données personnelles de ses adhérents résultant du fait d’une mauvaise manipulation par un membre actif, commise de bonne foi.

    \subsection{Usurpation d'identité}

		Toute usurpation d'identité, de quelque forme que ce soit, sera sévèrement sanctionnée. Cela comprend :

		\begin{itemize}
		    \item[\textbullet] les tentatives d'utilisation d'adresses IP autres que celles attribuées par l'association ;

		    \item[\textbullet] les tentatives de changement d'adresses physiques, ou adresse MAC ;

		    \item[\textbullet] toutes tentatives visant à masquer son identité sur le réseau géré par l'association ;

		    \item[\textbullet] l'utilisation d'une adresse physique (MAC) ou d'une adresse IP associée au compte d'un autre utilisateur, que ce soit pour bénéficier des services de l'association sans s'affranchir de la cotisation, pour échapper à une sanction, ou pour tout autre motif 
		\end{itemize}

\section{Informatique et Libertés}

    \subsection{Vie privée}


		L’association s’engage à tout mettre en œuvre en règle générale pour qu’il ne puisse être porté atteinte à la vie privée de ses adhérents. Tout non respect de cet engagement par un membre actif l’expose à des sanctions. L’association s’engage à signaler, sans exception, toute infraction avérée et manifeste à la vie privée au Procureur de la République.

    \subsection{Responsabilité}


		Sur le réseau interne, les utilisateurs sont identifiés de manière non équivoque par les adresses physiques correspondant aux cartes réseaux de leurs équipements personnels. Ces équipements doivent être enregistrés auprès de l'association avant la première utilisation, soit au moment de l'adhésion à l'association, soit par le biais de l'espace personnel sur le site web interne et ce, à raison de dix enregistrements maximum par personne. En cas d'utilisation justifiée de plus de machines, tout administrateur peut décider d'accepter des enregistrements supplémentaires. À chaque adresse physique est associée une adresse IP fournie par l'association pour un usage exclusivement personnel. Chaque adhérent est présumé responsable des faits et actions de ses appareils connectés au réseau. Une vigilance particulière est donc fortement conseillée aux adhérents quant à la mise à disposition à autrui de leurs équipements.

    \subsection{Confidentialité}

		Pour des raisons de protection juridique, l'association est tenue de conserver un certain nombre de données propres à chaque connexion depuis ou vers un utilisateur du réseau interne. Ces données comprennent :

		\begin{itemize}
 			\item[\textbullet] les informations permettant d'identifier l'utilisateur et l'équipement terminal utilisé ;
 			\item[\textbullet] les données permettant d'identifier le destinataire de la connexion.
 		\end{itemize}

		Ces données sont conservées pour une durée d'une année, dans une finalité de recherche, de constatation et de poursuite des infractions pénales et dans le seul but de permettre, en cas de besoin, la mise à disposition de ces données à l'autorité judiciaire.
		Ces données sont sous la responsabilité personnelle du Président de l'association.

\section{Fonctionnement et administration}

    \subsection{Administration du Rézoléo}
		                                        
		Le réseau est administré par des membres adhérents volontaires, éventuellement membres du Conseil d’Administration. Cette maintenance se fait à titre strictement bénévole. Aucune garantie ne peut être exigée quant au fonctionnement du réseau interne et des services associés. En particulier, l’association ne peut être tenue responsable :

		\begin{itemize}
			\item[\textbullet] d’éventuelles pertes de données que pourrait occasionner l’usage des services fournis par l’association ;
	 		\item[\textbullet] des dommages que pourrait causer une interruption de service. 
 		\end{itemize}
                                          
		Néanmoins, ces membres volontaires sont responsables de l’utilisation faite des accès qui leurs sont donnés à des fins d’administration du réseau. Il leur est ainsi formellement interdit d’utiliser ces accès pour :

		\begin{itemize}                                 
			 \item[\textbullet] accéder à des informations sur les adhérents à des fins personnelles ;
			 \item[\textbullet] utiliser les ressources du Rézoléo à des fins personnelles ;
			 \item[\textbullet] acquérir des accès à des services sans l’accord du Conseil d’Administration. 
		\end{itemize}
                            
		Les sanctions encourues sont les mêmes que pour les autres adhérents, en y incluant en surplus :

		\begin{itemize}                       
 			\item[\textbullet] la perte des accès qu’il possède ;
 			\item[\textbullet] le renvoi du Conseil d’Administration, le cas échéant, celui-ci statuant à la majorité qualifiée des deux tiers.
 		\end{itemize}
		Toute décision prise par un membre du Conseil d’Administration qui est contestée par un autre membre du Conseil d’Administration doit être votée par le Conseil d’Administration.\\

		\textbf{Président}

		Le Président est le responsable moral de l’association. Il est à ce titre habilité à agir auprès des tiers au nom de l’association. Il est garant, avec le Trésorier, de la santé de l’association.\\

		\textbf{Trésorier}

		Le Trésorier est responsable de la gestion comptable et financière de l’association, notamment du fonctionnement de ses comptes bancaires ainsi que de la saisie comptable. Il assure le recouvrement des cotisations et des ressources de toute nature de l’association. Il effectue les paiements et perçoit les recettes avec l’accord du Président.\\

		\textbf{Secrétaire}

		Le Secrétaire est responsable de la bonne tenue des réunions, ainsi que de leur compte-rendu. Il est responsable de la gestion administrative de l’association. Il est également habilité sous réserve du droit des tiers à délivrer tous les documents officiels du Rézoléo.\\

    \subsection{Modalités d'élection}

		Le Bureau est élu par les membres adhérents de l’association. Les membres souhaitant se présenter aux différents postes doivent émettre leur candidature une semaine à l’avance. Cette candidature peut être refusée par le Bureau en place. Le Bureau comporte obligatoirement un Président, un Secrétaire et un Trésorier. Si elles le jugent nécessaire, il pourra être choisi par chacune des parties prenantes du Bureau un adjoint parmi les autres membres adhérents afin de les aider dans leurs prérogatives. Il est possible de voter par procuration (par lettre ou par mail signé selon un protocole sécurisé).

    \subsection{Sanctions}

		En cas de non-respect d’une des dispositions de ce présent règlement, des sanctions peuvent être engagées envers un adhérent. Celles-ci sont prises par le Conseil d’Administration ou subsidiairement par le Bureau de l’association. L’adhérent peut contester la sanction appliquée, il devra alors présenter sa défense devant le Bureau de l’association qui pourra alors décider de révoquer, de modifier ou de conserver la sanction appliquée. Le cas échéant, la sanction décidée en premier lieu restera en application jusqu’à la défense de son cas par l’adhérent concerné.
		                                                        
		Ces sanctions comprennent :
	
		\begin{itemize}                          
 			\item[\textbullet] la suspension temporaire, pouvant aller d’un jour à un mois, de l’accès à certains services ;
 			\item[\textbullet] l’arrêt définitif de l’accès à certains services ;
 			\item[\textbullet] l’exclusion de l’association ;
 			\item[\textbullet] la limitation de l’accès à certains services.
 		\end{itemize}

    \subsection{Notes de frais}
 
		Les membres de l’association peuvent être amenés à engager des frais, notamment de déplacement,dans le cadre de leur investissement au sein de l’association. Ils peuvent se faire rembourser par l’association sur présentation d’une note de frais sur accord du Conseil d’Administration. Dans le cas où le montant est inférieur à 40\euro , le seul accord du Trésorier de l’association est nécessaire.

    \subsection{Protocole sécurisé}

		Un protocole sécurisé est un protocole cryptographique considéré comme sûr par le Bureau et adopté par l’Assemblée Générale.
		Le protocole convenant est OpenPGP normalisé par l’Internet Engineering Task Force (IETF). L’identité de la personne devra avoir été validée par un Responsable Technique.

\end{document}
