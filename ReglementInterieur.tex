\documentclass[12pt, a4paper]{article}

\usepackage{mathpazo}
\usepackage[french]{babel}
\usepackage[T1]{fontenc}
\usepackage[utf8]{inputenc}
\usepackage{array, multirow, tabularx}
\usepackage{graphicx}
\usepackage{multicol}

\usepackage{fontspec}
\setmainfont{Open Sans}[Path=./fonts/, Extension=.ttf,
  UprightFont=*-Regular,
  BoldFont=*-Bold,
  ItalicFont=*-Italic]

\begin{document}
	\title{Règlement intérieur}
	\author{Association Rézoléo}
	\date{08/07/2024}
	\maketitle

    \begin{center}
	   \textbf{Préambule}\\
          Le présent règlement a été ratifié par le Bureau le XX/07/2024

        \vspace{3cm}
        \includegraphics[scale=2]{rezoleo_logo.png}
          
    \end{center}

	\newpage

	\tableofcontents

	\newpage

	\section{Modalités d'adhésion et cotisation}

	\subsection{Cotisation}

	Les membres d'honneur ainsi que les Adhérents non connectés sont exonérés de cotisation.

	\bigskip

	Tout Adhérent non connecté tel que défini dans les statuts, ayant domicile à la
	résidence Léonard de Vinci (située au 400 Avenue Paul Langevin, 59650
	Villeneuve d'Ascq) et non précédemment radié des services de Rézoléo est
	susceptible de devenir Adhérent connecté de l'association.

	\bigskip

	Tout adhérent légalement incapable devra au préalable fournir l'autorisation écrite
	d'un responsable légal, qui s'engage auprès de l'association au nom de l'adhérent
	vis-à-vis du présent règlement et de tous les autres documents relatifs à l'utilisation
	des services du Rézoléo afin de pouvoir devenir adhérent connecté.

	\bigskip

	La cotisation de l'Adhérent connecté est fixée à compter du 20/08/2024 à 5€
	par mois, avec une ristourne fixée à 50€ pour une adhésion de 12 mois. Cette
	tarification est susceptible d'évoluer, sans qu'aucune évolution ne puisse
	avoir un effet sur les adhésions en cours lors du changement.

	\bigskip

	La qualité de membre connecté se perd notamment par fin sans renouvellement de
	la période de cotisation, qui entraîne la fin de tous les droits et
	prérogatives associées à ce type d'adhésion. Si l'ancien Adhérent connecté remplit
	toujours les conditions nécessaires telles que définies dans les statuts de l'association,
	il redevient automatiquement Adhérent non connecté de l'association.

	\bigskip

	Les membres du bureau sont autorisés s'ils le souhaitent à exonérer d'adhésion
	une personne quittant la résidence Léonard de Vinci au maximum un mois plus tard.

	\subsection{Moyens de paiement}

	Les seuls moyens de paiement acceptés par l'association sont :

    \bigskip

	\begin{itemize}
		\item[\textbullet] Les chèques ;

		\item[\textbullet] Les virements bancaires ;

		\item[\textbullet] Les espèces ;

		\item[\textbullet] Les cartes bancaires
	\end{itemize}

	\subsection{Remboursement de la cotisation}

	En cas de départ définitif de la résidence Léonard de Vinci, un adhérent
	connecté peut percevoir le remboursement des mois d'adhésion non entamés. Les
	mois obtenus via une ristourne ou à titre gracieux d'une quelconque manière que
	ce soit ne sont bien sûr pas remboursables.

	\bigskip

	Pour ce faire, l'adhérent envoie un courriel formulant sa demande à destination
	de Mme la Trésorière ou M. le Trésorier à l'adresse suivante : \textit{rezoleo@rezoleo.fr},
	auquel il joint la facture Rézoléo faisant foi de l'adhésion en cours, ainsi qu'un
	RIB sur lequel il souhaite se faire rembourser. Le remboursement sera alors effectué
    dans un délai raisonnable, exclusivement par virement bancaire.

	\subsection{Accès au réseau et services mis à disposition}

	Rézoléo s'efforce de rendre disponible les différents services mis à
	disposition à ses adhérents connectés et non connectés. La liste des services ainsi
	que les modalités d'utilisation de ces ressources sont déterminées au cas par cas
	par le bureau.

	\bigskip

	Néanmoins, un ou plusieurs de ces services peuvent être indisponibles à tout moment
	et sans préavis, que ce soit pour des raisons techniques ou à l'occasion d'évènements
	spéciaux par exemple. L'adhérent prend acte qu'aucun recours n'est possible en
	cas de défaillance ponctuelle des services proposés par Rézoléo.

	\bigskip

	De même, aucune garantie ne peut être exigée quant au fonctionnement du réseau
	interne et des services associés. En particulier, l'association ne peut être tenue
	responsable d'éventuelles pertes de données que pourraient occasionner l'usage
	des services fournis par l'association ainsi que des dommages autres que pourraient
	causer une interruption de service.

	\subsection{Clubs et associations}

	Les clubs et commissions de Centrale Lille Associations ainsi que les
	associations Loi 1901 présentes à la Résidence Léonard de Vinci et à l'École
	Centrale de Lille ainsi que l'AGR peuvent
	obtenir à titre gracieux un accès aux services habituellement réservés aux
	membres connectés, sous réserve de signature charte de bonne utilisation du réseau
	par le responsable légal de cette entité, renouvelable à chaque changement de
	responsable légal. Rézoléo se réserve le droit de suspendre l'accès à ces services
	dans le cas où la charte ne serait pas signée ou respectée.

	\bigskip

	Les personnes morales accèdent au réseau avec les mêmes droits et devoirs que
	les personnes physiques, tels que définis dans les statuts et dans le présent réglement.

	\section{Usage spécifique des services de l'Association}

	\subsection{Membre actif}

	Les membres du bureau votent à la majorité la nomination parmi les adhérents non
	connectés ou les membres d'honneurs des "Membres actifs", dont le/la
	secrétaire de l'association tient une liste, qui peut être consultée par l'École
	Centrale de Lille, l'AGR, ainsi que tout personne dépositaire de l'autorité publique,
	par le truchement d'un courriel envoyé à \textit{rezoleo@rezoleo.fr}. Le nombre de membres
	actifs est illimité. Les membres du bureau peuvent se nommer eux-mêmes membres
	actifs.

	\bigskip

	Le statut de membre actif peut conférer notamment le droit à un accès physique
	et/ou distant aux logiciels et aux serveurs de l'association, à leur
	administration, à l'encaissement des cotisations ainsi qu'à la gestion des adhésions,
	ainsi qu'un accès aux données relatives aux adhérents, dans le respect des normes
	en vigueur.

	\bigskip

	Le statut de membre actif est bénévole.

	\subsection{Usage des droits conférés à un membre actif}

	Les membres actifs sont responsables de leurs actions et en répondent auprès du
    secrétaire, et en cas de besoin, auprès du président, responsable
	moral de l'association.

	\bigskip

	Tout usage par un membre actif de ses droits dans une situation inappropriée,
	ou hors du cadre de ses missions pour l'association, est interdit. Un tel
	usage pourra être sanctionné sans délai par un retrait des attributions en
	question, dans le respect des statuts et du présent règlement. Selon la nature
	de l'infraction, des mesures juridiques pourront être prises par l'association.

	\bigskip

	Il leur est notamment formellement interdit d'utiliser leurs accès pour
	accéder à des informations sur les adhérents à des fins personnelles ou
	utiliser les ressources du Rézoléo à des fins personnelles hors du cadre prévu
	par l'association.

	\subsection{Responsabilité des membres actifs}

	L'association assure le travail de ses membres actifs, leur matériel, et leur personne,
	dans le cadre de l'exercice de leurs fonctions. Elle prend la responsabilité de
	toute altération de la connectivité à Internet ou dommage matériel, logiciel, ou
	relatif aux données personnelles de ses adhérents résultant du fait d'erreurs de
	bonne foi commises par un membre actif.

	L'association ne saurait tolérer quelconque dénigrement ou agression verbale ou
	physique à l'encontre de ses membres actifs dans le cadre de l'exercice de leur
	fonction, et se réserve le droit d'engager toutes les poursuites judiciaires
	qu'elle jugera nécessaires le cas échéant.

	\subsection{Usurpation d'identité}

	Toute manoeuvre s'apparentant à une usurpation d'identité, de quelque forme
	que ce soit, sera sévèrement sanctionnée. Cela comprend, mais ne se limite pas
	à :

	\bigskip

	\begin{itemize}
		\item[\textbullet] les tentatives d'utilisation d'adresses IP autres que celles
			attribuées par l'association ;

		\item[\textbullet] les tentatives de changement d'adresses physiques, ou d'adresses
			MAC ;

		\item[\textbullet] les tentatives visant à masquer son identité sur le réseau
			géré par l'association ;

		\item[\textbullet] l'utilisation d'une adresse physique (MAC) ou d'une adresse
			IP associée au compte d'un autre utilisateur, que ce soit pour bénéficier
			des services de l'association sans s'affranchir de la cotisation, pour échapper
			à une sanction, ou pour tout autre motif ;

		\item[\textbullet] le partage de compte pour faire bénéficier à un tiers du service.
	\end{itemize}

	\bigskip

	Outre la tentative de fraude constituant de telles infractions, Rézoléo attire
	l'attention de l'adhérent sur le fait qu'un partage de compte expose le
	contrevenant à toutes les poursuites légales résultant de l'utilisation de son
	compte, même s'il n'en est pas l'auteur. Au cas où les autorités viendraient
	demander des comptes à l'association en tant que fournisseur internet officiel,
	celle-ci est dans l'incapacité de désigner une autre personne que le propriétaire
	du compte.

	\section{Informatique et Libertés}

	\subsection{Vie privée}

	L'association s'engage à tout mettre en œuvre en règle générale pour qu'il ne puisse
	être porté atteinte à la vie privée de ses adhérents. Tout non respect de cet
	engagement par un membre actif l'expose à des sanctions. L'association s'engage
	à signaler, sans exception, toute infraction avérée et manifeste à la vie privée
	aux autorités compétentes.

	\subsection{Responsabilité}

	Sur le réseau interne, les utilisateurs sont identifiés de manière non équivoque
	par les adresses physiques correspondant aux cartes réseaux de leurs équipements
	personnels. Ces équipements doivent être enregistrés auprès de l'association
	avant la première utilisation, soit automatiquement quand c'est possible, soit
	au moment de l'adhésion à l'association, soit par le biais de l'espace personnel
	sur le site web interne et ce, à raison de quatre enregistrements maximum par
	personne. En cas d'utilisation justifiée de plus de machines, des
	enregistrements supplémentaires pourront être effectués.

	\bigskip

	À chaque adresse physique est associée une adresse IP fournie par l'association
	pour un usage exclusivement personnel. Chaque adhérent est responsable des
	faits et actions de ses appareils connectés au réseau. Une vigilance particulière
	est donc fortement conseillée aux adhérents quant à la mise à disposition de leurs
	équipements, ou de la sécurité de leur mots de passe.

	\subsection{Confidentialité}

	Pour des raisons juridiques et légales, l'association est susceptible de
	conserver un certain nombre de données propres à chaque connexion depuis ou
	vers un utilisateur du réseau interne. Ces données comprennent :

	\bigskip

	\begin{itemize}
		\item[\textbullet] les informations permettant d'identifier l'utilisateur et
			l'équipement terminal utilisé (Nom, prénom, adresse mail, chambre, adresse
			MAC, IP, port) ;

		\item[\textbullet] les données permettant d'identifier le destinataire de la
			connexion ;

		\item[\textbullet] Les journaux de bords "logs" pendant une durée d'un an ; 

		\item[\textbullet] Toutes les factures pendant une durée de 10 ans.
	\end{itemize}

	\bigskip

	Ces données sont conservées uniquement dans une finalité de recherche, de
	constatation et de poursuite des infractions pénales et dans le seul but de
	permettre, en cas de besoin, la mise à disposition de ces données à l'autorité
	judiciaire. Ces données sont placées sous la responsabilité du responsable moral
	de l'association.

	\bigskip

	Pour toute réclamation quant à l'utilisation et le stockage des données
	personnelles, merci d'écrire un courriel à \textit{rezoleo@rezoleo.fr}.

	\section{Fonctionnement et administration}

	\subsection{Administration de l'association}

	Comme défini dans les statuts, Rézoléo est administré par 3 personnes
	physiques distinctes formant le Bureau : le Président ou la Présidente, Le Trésorier
	ou la Trésorière, le Secrétaire ou la Secrétaire, tous trois dûment déclarés en
	Préfecture.

	\subsection{Administration technique}

	Par mesure de sécurité, en cas d'intrusion dans les systèmes de l'association,
	d'actions menaçant l'intégrité des systèmes du fait de n'importe quel membre
	de l'association ou bien d'une action extérieure, il existe un certain nombre de
	clés informatiques détenues par autant de personnes différentes permettant la récupération
	des accès administrateurs aux systèmes du Rézoléo, un certain nombre de ces
	clés étant nécessaire simultanément pour activer la procédure de récupération.

	\bigskip

	Toute utilisation d'urgence de ce mode de récupération doit s'accompagner d'une
	communication aux adhérents expliquant les raisons ayant poussé à une telle utilisation.

	\subsection{Modalités d'élection}

	Les modalités d'élection définies dans les statuts s'appliquent et prévalent.

    \bigskip

	Les procurations doivent être présentées à l'écrit et portées à la
	connaissance de l'assemblée générale au début des élections. Pour chaque vote,
	il est possible pour chaque voix délibérative de voter soit pour un candidat, soit
	de s'abstenir.

	\bigskip

	L'élection d'un nouveau bureau pour une durée d'un an suit dans l'ordre du jour d'une assemblée
	générale la démission des trois membres du bureau, potentiellement à l'écrit en
	cas d'absence. Les élections se font à main levée, sauf si au moins une des personnes
	ayant une voix délibérative ou consultative à l'assemblée s'y oppose.

	\bigskip

	Pour chaque poste, la liste des candidats légitimes est portée à la connaissance
	des adhérents présents à l'assemblée générale. Il est accordé à chaque
	candidat cinq minutes afin de prononcer un discours, suivi par au maximum cinq
	minutes de questions de l'assemblée.

	\bigskip

	Le scrutin prend place, les candidats pouvant quitter la pièce s'ils le souhaitent.
	Le candidat ayant obtenu le plus de voix l'emporte. En cas d'égalité, le vote
	majoritaire des trois membres du bureau sortant est décisif. En cas de persistance
	d'égalité entre trois candidats malgré cela, le vote du président sortant est
	décisif. Le résultat de l'élection du poste est alors annoncé et inscrit au procès
	verbal.

	\bigskip

	Les élections prennent place dans l'ordre suivant : Président, puis Trésorier,
	puis Secrétaire. En cas de non élection, il est possible de se présenter au
	poste suivant.

	\subsection{Sanctions}

	En cas de non-respect d'une des dispositions de ce présent règlement, des
	sanctions peuvent être engagées envers un adhérent. Celles-ci sont prises par le
	Bureau de l'association. L'adhérent peut contester la sanction appliquée, il devra
	alors présenter sa défense devant le Bureau de l'association qui pourra alors décider
	de révoquer, de modifier ou de conserver la sanction appliquée, sans que cette
	contestation ne revete un caractère suspensif vis-à-vis de la sanction.

	\bigskip

	Ces sanctions comprennent :

	\bigskip

	\begin{itemize}
		\item[\textbullet] la suspension temporaire (pouvant aller d'un jour à trois
			mois) ou l'arrêt définitif de l'accès à certains services ;

		\item[\textbullet] l'exclusion de l'association ;

		\item[\textbullet] la limitation de l'accès à certains services.
	\end{itemize}

	\subsection{Notes de frais}

	Les membres de l'association peuvent être amenés à engager des frais,
	notamment de déplacement,dans le cadre de leur investissement au sein de l'association.
	Ils peuvent demander à se faire rembourser par l'association sur présentation d'une
	note de frais et d'un justificatif. Dans le cas où le montant est inférieur à
	50€ , le seul accord du Trésorier de l'association est nécessaire. Toutefois,
	dans tous les cas, le demandeur ne peut pas valider seul une note de frais.

	\subsection{Protocole sécurisé}

	Un protocole sécurisé est un protocole cryptographique considéré comme sûr par
	le Bureau et adopté par l'Assemblée Générale. Le protocole choisi est OpenPGP
	normalisé par l'Internet Engineering Task Force (IETF). L'identité de la
	personne devra avoir été validée par un Responsable Technique.

    \subsection{Modification du Règlement intérieur}

    Le présent règlement peut être modifié sans préavis et entre en vigueur à partir
    du moment où il est ratifié par l'ensemble des membres du bureau. Il appartient
    aux adhérents de le consulter régulièrement pour prendre connaissance des
    éventuelles modifications.
\end{document}
